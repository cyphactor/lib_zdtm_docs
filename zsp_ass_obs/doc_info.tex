%
% Copyright (c) 2003, 2004, 2005 Andrew De Ponte.
% Permission is granted to copy, distribute and/or modify this document
% under the terms of the GNU Free Documentation License, Version 1.2
% or any later version published by the Free Software Foundation;
% with the Invariant Sections being Front-Cover Texts, with the
% Front-Cover Texts being abstract. A copy of the license is included
% in the section entitled "GNU Free Documentation License".
%

%
% FileName:     doc_info.tex
% Author:       Andrew De Ponte
% E-Mail:       cyphactor@socal.rr.com
% AIM:          HUNNYnNUTTS
% Description:  This document is a LaTeX document containing the
%               document information for the assumptions and observations
%               document about the synchronization
%               protocol used by Sharp's Zaurus SL-5600 v1.0 ROM and
%               Sharp's Zaurus SL-5500 v3.10 ROM.
%



\part{Document Information}

\section{Version Information}

    \subsection{Does this document have a version?}
    
    This document will not be given a version due to the fact that I do not know the
    protocol and that I am making assumptions about the protocol. When this
    document has been further compiled and it is based on testing, I may
    version the document in which case the version number will be trailing the
    \emph{Title} of the document.

    \subsection{How to distinguish different versions?}

    The method to distinguish whether this document is a newer or
    older version is by checking the production date right below the title.
    If the production date is more recent than the production date on your
    copy of this document you should use the one with the newer production
    date.

    \subsection{What are the rules on releases?}

    The most recent release will \emph{not} always contain more
    information than the last release. It is possible that some
    of the information in the older release was incorrect and was removed.
    It is also possible that the two releases look almost the same because
    only a portion was modified.
    Despite all of these possibilities, the common practice will
    be to add more information on top of existing
    data from the previous releases of the document.

% Here I describe why this document and the effort behind it
% is needed.
\section{Why does this document exist?}

    \subsection{Sharp's Decision}

    This document and the effort behind it is needed because Sharp has
    decided that they are only going to support Microsoft Windows users
    when it comes to synchronizing the Zaurus SL-5600, leaving their
    trusty Linux users out in cold. I read some where that this was due
    to the fact that Sharp develops their Zaurus product for corporations and
    that most of the corporations they develop the Zaurus for use Microsoft
    Windows.

    \subsection{Obtaining the Specifications}

    I contacted Sharp and asked them if I could possibly obtain a copy of
    the specifications for the synchronization protocol since they had
    decided that they were not going to create a Linux application for
    synchronization. They responded by telling me to go to their
    developers site and talk with other developers to figure it out. After
    looking all over the web and talking to all kinds of people I
    decided that no specifications were available for this protocol.

    \subsection{The Chosen Solution}

    After my failure at the quest for the protocol specifications, I decided
    that it was up to me to come up with a solution. After contemplating
    possible solutions for about a week I decided that my decision was to
    reverse engineer the protocol. I chose this solution because I think that
    the Zaurus SL-5600 users should be able to synchronize their Zaurus with
    their Linux box without having to install any special software on their
    Zaurus.

    \subsection{The Purpose}

    This document contains the assumptions and observations of the on going
    reverse engineering of the Zaurus SL-5600 synchronization protocol so that
    developers may use the information to develop a synchronization software
    application for Linux that will allow Linux users to synchronize their
    Zaurus SL-5600, right out of the box.
